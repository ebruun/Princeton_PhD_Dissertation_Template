% -------------------------
% ABSTRACT
% -------------------------

% \abstracttext{
%     Robotic automation has the potential to revolutionize the Architecture, Engineering, and Construction (AEC) industry by improving task efficiency and facilitating novel construction processes for complex structures. Cooperative Robotic Fabrication (CRF) involves orchestrating robots to work together on intricate construction tasks that exceed the capabilities of individual robots. This dissertation focuses on developing and demonstrating CRF methods for assembling, disassembling, and reusing structures, with the overarching goal of reducing resource consumption and promoting sustainability in construction.

%     In this dissertation, four physical case studies are used to demonstrate the practical application of CRF in diverse construction scenarios, spanning various materials, scales, and discrete element structural typologies. A central innovation demonstrated throughout all the case studies is using CRF setups to maintain structural stability during construction without relying on external temporary supports or scaffolding structures, achieved through carefully coordinating the sequence of actions taken by teams of two or more robots. The case studies progress from simple assembly-only applications to more complex structural disassembly and reuse applications.
    
%     In the first case study, a novel bottom-up design framework is introduced, where two cooperating robots actively contribute to designing branching spatial structures in real-time collaboration with human decision-making. In the second case study,  cooperative fabrication methods are utilized for constructing complex masonry structures, where two or three robots are sequenced to ensure the stability of the central arch without requiring external scaffolding. In the third case study, a fabrication-informed approach is presented for the design of space frame structures, leveraging rigidity theory to design structures that are specifically planned to be assembled and disassembled in a scaffold-free manner using two cooperating robots. In the fourth case study, a setup with three robots is used for the scaffold-free disassembly and reuse of an existing prototype structure, aligning with circular economy principles by minimizing material waste and enabling reuse.
    
%     The research in this dissertation demonstrates that multiple robots working together can be applied in different ways across a variety of construction scenarios. As the first of their kind, the case studies demonstrating the use of CRF setups to perform scaffold-free assembly, disassembly, and reuse of spanning structures are examples of novel construction methodologies within the AEC industry.
% }

\abstracttext{
Robotic automation has the potential to revolutionize the Architecture, Engineering, and Construction (AEC) industry by enabling novel construction processes for geometrically complex structures. Cooperative Robotic Fabrication (CRF) involves coordinating multiple robots to work together on construction tasks beyond the capabilities of individual robots. This dissertation explores the use of CRF methods for assembling, disassembling, and reusing discrete element structures. The key innovation lies in developing and physically demonstrating computational methods that enable teams of robotic arms to work together. These methods allow the robotic arms to precisely coordinate their actions to ensure structural stability is maintained during construction without needing external supports or scaffolding.

The dissertation begins with assembly-only applications and extends to more complex tasks involving structural disassembly and reuse. It starts with a bottom-up human-robot design framework where two cooperating robots help design branching spatial structures in pseudo real-time, using path-planning constraints in collaboration with human decision-making. Next, CRF is demonstrated for constructing spanning masonry structures. Here, the structure is modeled as a lumped spring system, and multiple robots are sequenced in either a sequential, cantilever, or optimized manner to ensure the stability of the central arch during construction without external scaffolding. Next, a resource-informed approach is demonstrated for designing space frame structures. This approach uses rigidity theory and Henneberg planar graph assembly steps to sequentially design and build rigid space frame structures that are specifically intended for scaffold-free assembly and disassembly using a CRF setup. Finally, CRF is applied to the scaffold-free disassembly and reuse of an existing structure, aligning with circular economy principles by minimizing material waste and enabling reuse. A novel graph-based method utilizing the concept of structural support hierarchy is developed to isolate affected members within the structure. By unrolling the resulting subgraphs, this method generates sequences for removing members without compromising the integrity of the remaining structure.

This dissertation demonstrates that multiple robots working together can be utilized in previously unexplored construction scenarios, involving diverse materials, scales, and discrete element structural typologies. These practical demonstrations introduce novel cooperative robotic scaffold-free assembly, disassembly, and reuse methods to the AEC industry.
}